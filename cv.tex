\documentclass[11pt,a4paper]{moderncv}

% moderncv themes
\moderncvtheme[green]{classic}

% character encoding
\usepackage[utf8]{inputenc}
% adjust the page margins
\usepackage[scale=0.8]{geometry}

\recomputelengths

% personal data
\firstname{Igor}
\familyname{Moreno Santos}
\title{Postdoc at Università della Svizzera italiana (USI)}
\address{via Trevano 39}{6900 Lugano}{Switzerland}
\mobile{+41 78 898 5851}
\email{igormoreno@gmail.com}                               
\homepage{igormoreno.com}
%\photo[64pt]{jdoe_picture}
\quote{``Progamming can be Mathematics'' - Conal Elliott}

\begin{document}

\maketitle

\section{Summary}
\cvitem{}{I'm currently a postdoc at Università della Svizzera italiana (USI) in Lugano (Switzerland)
working at the \textcolor{cyan}{\href{https://luce.si.usi.ch/}{LuCE research lab}}, lead by professor \textcolor{cyan}{\href{https://www.inf.usi.ch/faculty/hauswirth/}{Matthias Hauswirth}}.
My research interests are in Programming Language theory, design, and implementation as well as in Computer Science Education.}

\section{Education}
\cventry{2017 -- 2023}{Ph.D. in Computer Science}{Università della Svizzera italiana (USI)}{Lugano, Switzerland}
{}
{LuCE research lab, advised by professor Matthias Hauswirth. \\
My thesis leveraged techniques from Programming Language Theory to establish a theoretical foundation for Notional Machines, pedagogic devices widely used in Computer Science Education. I demonstrated its use for reasoning about Notional Machines and their relationship with the aspect of programs under their focus.}

%The topic of my PhD was on establishing a theoretical foundation for Notional Machines. A Notional Machine is a pedagogic device that abstracts away details of the semantics of a programming language to focus on some aspects of interest. Despite being widely used in Computer Science Education, neither Notional Machines nor their relationship with the aspect of programs they represent are formaly defined. That's important because it's ultimately how we can determine if a Notional Machine is correct/consistent with the aspect under its focus. To address this problem, I first introduced a formal definition of soundness for Notional Machines. The definition is based on the construction of a commutative diagram that relates the Notional Machine and the aspect of the programming language the Notional Machine is focused on, a formalism that amounts to an application of Bisimulation theory. Leveraging this formalism, I show how one can systematically construct sound Notional Machines and reveal potential inconsistencies in existing Notional Machines. 

%My thesis leveraged techniques from Programming Language Theory to establish a theoretical foundation for Notional Machines, pedagogic devices widely used in Computer Science Education. I then showed how one can use it to reason about a Notional Machine and its relationship with the aspect of programs under its focus.

\cventry{2014 -- 2017}{Master's degree in Computer Science}{Università della Svizzera italiana (USI)}{Lugano, Switzerland}
{}
{Thesis advised by professor Nate Nystrom. \\
The thesis focused on the design and implementation of backward functions
for a general-purpose functional programming language with mixfix functions.
The combination of these features provides a single underlying mechanism to
express many traditionally disparately implemented constructs:
operators,
control flow statements,
pattern matching and views,
automatic solution of simple equations and Boolean formulas,
and for-comprehensions.}

% Topics:
% - continued studying with a dictation software
% - focused on programming languages
% - thesis with Nate



% - thesis with Nate


% Topics:


% - thesis with Nate


% Topics:
\cventry{2012 -- 2014}{Master's degree in Computer Science (incomplete)}{Ecole polytechnique fédérale de Lausanne (EPFL)}{Lausanne, Switzerland}
{}
{I successfully completed several courses on Programming Languages theory and implementation, Algorithms and Distributed Systems but unfortunately had to pause the program after developing severe epicondylitis and decided to continue at USI.
Most notably, the course Foundations of Software by Martin Odersky,
based on Benjamin Pierce’s Types And Programming Languages (TAPL),
solidified my intuition that programming languages themselves could be mathematical entities
and was the main motivation for me to pursue a PhD in Programming Languages.}


% Topics:
% - took several important courses but had health issues and had to pause the program
% - Foundations of Software was a turning point... discovered the beauty of programming languages

% Taking the course Foundations of Software by Martin Odersky, based on Benjamin Pierce’s Types And Programming Languages (the TAPL book), was a water-shattering experience for me. I remember when saw the Curry-Howard correspondence and thought to myself: “how did I never hear about this before?”. What amazed me wasn’t only proving properties about programming languages but that programming languages themselves could be mathematical entities. It also confirmed my intuition that some “things” in programming languages were more fundamental than others.

\cventry{2010 -- 2012}{Bachelor's degree in Computer Science}{Università della Svizzera italiana (USI)}{Lugano, Switzerland}
{}
{The experience of working at the ILO in Geneva broadened my horizons and motivated me to study abroad. I managed to have part of the credits I had obtained accepted at USI.}

% Topics:
% - after geneva, transfered my credits to USI
% - thesis with Binder

\cventry{2003 -- 2007 2009 -- 2010}{Bachelor's degree in Computer Science (incomplete)}{Universidade de Brasília (UnB)}{Brasília, Brazil}
{}
% Topics:
% - studying and working
% - learning a lot
% - working contacts lead me to internship in geneva
% - after experience in geneva I decided to move to switzerland so i transfered my credits to USI
{Early on in my studies I was already working as a software developer.
The private sector environment was very stimulating and I learned a lot
but my increasing dedication to work delayed my graduation substantially.}


\section{Academic Employment}
\cventry{2023 --}{Postdoc}{Università della Svizzera italiana (USI)}{Lugano, Switzerland}{}{LuCE research lab, lead by professor Matthias Hauswirth.}

\section{Publications}
% Luca Chiodini, Igor Moreno Santos, Matthias Hauswirth. Expressions in Java: Essential, Prevalent, Neglected?. SPLASH-E '22. [PDF] [DOI]
\cventry{\footnotesize{SPLASH-E '22 [\textcolor{cyan}{\href{https://dl.acm.org/doi/10.1145/3563767.3568131}{DOI}}]}}
{Expressions in Java: Essential, Prevalent, Neglected?}
{}{}{}
{Luca Chiodini, Igor Moreno Santos, Matthias Hauswirth}
% Luca Chiodini, Igor Moreno Santos, Andrea Gallidabino, Anya Tafliovich, André L. Santos, Matthias Hauswirth. A Curated Inventory of Programming Language Misconceptions. ITiCSE '21. [PDF] [DOI]
\cventry{\footnotesize{ITiCSE '21 [\textcolor{cyan}{\href{https://dl.acm.org/doi/10.1145/3430665.3456343}{DOI}}]}}
{A Curated Inventory of Programming Language Misconceptions}
{}{}{}
{Luca Chiodini, Igor Moreno Santos, Andrea Gallidabino, Anya Tafliovich, André L. Santos, Matthias Hauswirth}
\cventry{\footnotesize{SPLASH-E '19 [\textcolor{cyan}{\href{https://dl.acm.org/doi/abs/10.1145/3358711.3361628}{DOI}}]}}
{Experiences in bridging from functional to object-oriented programming}
{}{}{}
{Igor Moreno Santos, Matthias Hauswirth, Nate Nystrom}
\cventry{2008 -- 2017}{Software Engineer}{Freelance}{}{}{
Freelancing in software development.}
%\cventry{Sept 2009 -- Apr 2010}{Consultant}{International Labour Organization (ILO)}{}{}{
\cventry{2010}{Consultant}{International Labour Organization (ILO)}{}{}{
Reviewed, improved, and updated the ILO Employment Sector portal.}
%\cventry{June 2009 -- Mar 2010}{Software Engineer}{Topológica}{}{}{
\cventry{2009 -- 2010}{Software Engineer}{Topológica}{}{}{
Development of distributed systems and software interfaces.}
%\cventry{Jan 2009 -- May 2009}{Software Engineer}{Flow e-commerce}{}{}{
\cventry{2009}{Software Engineer}{Flow e-commerce}{}{}{
Web application development using Symfony PHP Framework.}
%\cventry{Nov 2007 -- July 2008}{Tester}{International Labour Organization (ILO)}{}{}{
\cventry{2007 -- 2008}{Tester}{International Labour Organization (ILO)}{}{}{
Development of automated tests and CMS template parser.}
%\cventry{July 2005 -- Oct 2007}{Software Engineer}{AgênciaClick Isobar}{}{}{
\cventry{2005 -- 2007}{Software Engineer}{AgênciaClick Isobar}{}{}{
Development and analysis of object-oriented systems, web applications.}

\section{Languages}
\cvlanguage{Portuguese}{Native}{}
\cvlanguage{English}{Proficient}{}
\cvlanguage{Italian}{B1-B2}{}
\cvlanguage{French}{A2}{}
\cvlanguage{Spanish}{A2}{}
\cvlanguage{German}{A1}{}

\end{document}

